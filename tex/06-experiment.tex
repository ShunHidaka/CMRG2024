

\textcolor{structure.fg}{\textbullet} \ 使用する行列\\
	 1. VCNT90000:$90000$次実対称行列,$\sigma_k = 0.001 \exp(\frac{2 \pi \i}{50}(k-0.5)),\ (M=50)$\\
	 2. VCNT10800h:$10800$次エルミート行列,$\sigma_k =(0.4+\frac{k}{1000}) + 0.001\i,\ (M=1001)$\\
\textcolor{structure.fg}{\textbullet} \ 計算環境\\
	 富岳 A64FX, 48 cores, 2.0 GHz 1node, Fujitsu C Compiler\\
\textcolor{structure.fg}{\textbullet} \ 実験内容\\
	 1. Generalized shifted COCG法との比較(相対残差$\frac{\|\vb{b}-(A+\sigma_{k}B)\vb{x}_{n}^{(k)}\|_2}{\|\vb{b}\|_2}$,実行時間)\\
	 2. 内部反復の精度と外部反復の反復回数・真の相対残差の関係の調査


%
% 2. について 0.001だと小さすぎて収束しないかも、0.01なら上手くいくかも
%


\begin{comment}

\begin{itemize}
	\item 使用する行列\\
		1. VCNT90000:$90000$次実対称行列\\
		2. VCNT10800h:$10800$次エルミート行列
	\item 計算環境\\
		富岳 A64FX, 48 cores, 2.0 GHz 1node, Fujitsu C Compiler
	\item 実験内容\\
		1. Generalized shifted COCG法との比較(相対残差$\frac{\|\vb{b}-(A+\sigma_{k}B)\vb{x}_{n}^{(k)}\|_2}{\|\vb{b}\|_2}$,実行時間)\\
		2. 内部反復の精度と外部反復の反復回数・真の相対残差の関係の調査
\end{itemize}

\end{comment}

%		VCNT225000