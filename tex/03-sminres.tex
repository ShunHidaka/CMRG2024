

\begin{enumerate}
	\item $A$に対するLanczos過程でKrylov部分空間$\KS{n}{A}{\vb{b}}$の正規直交基底を構成
		\begin{align*}
			AV_{n} = V_{n+1}\widehat{T}_{n},\ V_{n}=[\vb{v}_1 \ \cdots \ \vb{v}_n],\ \widehat{T}_{n} =
			{\small
			\begin{bNiceMatrix}[nullify-dots, columns-width=1.5em, cell-space-top-limit=5pt, cell-space-bottom-limit=5pt]
				\alpha_{1}	& \beta_{1}	&		&			\\
				\beta_{1}	& \Ddots	& \Ddots	& 			\\
    						& \Ddots	&		& \beta_{n-1}	\\
    						&		&		& \alpha_{n}		\\
    						&		&		& \beta_{n}		\\
			\end{bNiceMatrix}\text{:三重対角行列}
			}
		\end{align*}
		$AV_{n} = V_{n+1}\widehat{T}_{n}$は$A$のLanczos分解という\\
		\myitem シフト不変性により$\KS{n}{A+\sigma_{k}I}{\vb{b}}$の正規直交基底でもある
	\item $\widehat{T}_{n}^{(k)} = \widehat{T}_{n} + \sigma_{k}\mqty[I \\ \vb{0}^\top]$とおく\\
		\myitem $(A+\sigma_{k}I)V_{n}=V_{n+1}\widehat{T}_{n}^{(k)}$が成り立つ
	\item $\widehat{T}_{n}^{(k)}$のQR分解を計算する($T_{n}^{(k)} = Q_n R_n$)
	\item $\vb{y}_{n}^{(k)} = \norm{\vb{b}}_2 R_{n}^{-1} Q_{n}^{\htop} \vb{e}_{1}$を求める
		\begin{align}
			\| \vb{r}_n^{(k)} \|_2
			= \| \vb{b} - (A+\sigma_{k}I)\vb{x}_{n} \|_2
			&= \Bigl\| V_{n+1} \left( \|\vb{b}\|_2 \vb{e}_1 - \widehat{T}_{n}^{(k)} \vb{y}_{n}^{(k)} \right) \Bigr\|_2 \notag\\
			&= \Bigl\| \|\vb{b}\|_2 \vb{e}_1 - \widehat{T}_{n}^{(k)} \vb{y}_{n}^{(k)} \Bigr\|_2
			\label{03-siki}
		\end{align}
	\item 最小残差解 $\vb{x}_{n}^{(k)} = V_{n} \vb{y}_{n}^{(k)}$を求める
\end{enumerate}
\begin{itemize}
	\item \textcolor{red}{残差の単調減少性}と\textcolor{red}{無破綻性}を持つ
%	\item 漸化式で計算することで3本の基底ベクトルと補助ベクトルを保持するだけで済む
	\item 漸化式で計算することで効率的に計算できる
\end{itemize}


% シフト不変性により$V_{n}$は$\KS{n}{A+\sigma_k}{\vb{b}}$の正規直交基底でもある
%	\item[{\footnotesize \textcolor{black}{\textbullet}}] a

