

第$n$反復での残差の$B^{-1}$--ノルムおよびその最小残差解は式\eqref{05-minres-solution}となる
\begin{align}
	\|\vb{r}_{n}^{(k)}\|_{B^{-1}} = \|\vb{b} - (A + \sigma_{k}B)\vb{x}_{n}\|_{B^{-1}}
		&= \|\vb{b} - (A + \sigma B)W_{n}\vb{z}_{n} \|_{B^{-1}} \notag \\
		&= \left\| \norm{\vb{b}}_{B^{-1}}\vb{e}_1 - \hat{T}_{n}^{(k)} \vb{z}_{n} \right\|_2
		\label{05-Binv-residual}\\
	&\hspace*{-17.5cm}\Rightarrow \ \vb{x}_{n}^{(k)} = \|\vb{b}\|_{B^{-1}} W_{n} R_{n}^{-1} Q_{n}^{\htop} \vb{e}_1
	\label{05-minres-solution}
\end{align}
式\eqref{05-Binv-residual}は式\eqref{03-siki}とノルム$\|\vb{b}\|_2$および基底$V_{j}$を除いて一致した形をしている\\
\myitem 同様の漸化式が導かれる(Lanczos過程を一般化Lanczos過程に置き換える)\\
\myitem 残差の\textcolor{red}{$B^{-1}$--ノルム}の単調減少性と\textcolor{red}{無破綻性}を持つ