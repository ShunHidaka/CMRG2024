

\begin{itemize}
	\item (標準)シフト線形方程式
		\begin{align}
			(A + \sigma_{k}I)\vb{x}^{(k)} &= \vb{b}, \qquad (k=1,\dots,M).
		\end{align}
		Krylov部分空間のシフト不変性$\mathcal{K}(A+\sigma_kI, \vb{b}) = \mathcal{K}(A, \vb{b})$を持つ
		\begin{itemize}
			\item 利用した効率的な解法が存在(e.g. shifted MINRES法,shifted COCG法)
		\end{itemize}
	\item 一般化シフト線形方程式
		\begin{align}
			(A + \sigma_{k}B)\vb{x}^{(k)} &= \vb{b}, \qquad (k=1,\dots,M).
		\end{align}
		一般化固有値問題に対するSakurai-Sugiura法で現れる\\
		Krylov部分空間のシフト不変性$\mathcal{K}(A+\sigma_kB, \vb{b}) = \mathcal{K}(A, \vb{b})$を持たない\\
		既存手法としてGeneralized shifted COCG法($A+\sigma_kB$が複素対称)\cite{ref-SogabeT-2010}
\end{itemize}
%\vspace{0.5\baselineskip}
% 本発表では$A, B$がエルミート行列で,$\sigma_k$が複素数である場合を扱う.